\section{Информационная безопасность}

Так как приложение хранит пользовательские данные, то необходимо предупреждать пользователей о способах хранения
их данных как пользовательских. Для этого было написана политика обработки персональных данных.

Один из пунктов политики звучит следующим образом: ``По условиям настоящего документа пользователь соглашается с тем фактом, что все его данные будут храниться с использованием облачных технологий на серверах, расположенных за пределами территории РФ.
Тем не менее, эти данные защищаются от кражи или утери средствами информационной безопасности как Администрации, так и сторонних облачных сервисов хранения данных.''

Также были написаны специальные промежуточные слои, которые не позволяют запрашивать данные с сервера с любого устройства.
Для AJAX-запросов была разработана политка CORS(Cross-origin resource sharing) для того чтобы предотвратить лишние запросы к серверу с неразрешённых доменов.

Для хранения паролей используется шифрование sha256, а в базе они храняться в виде хеша. Для аутенфикации/авторизации также используется стандарт JWT.