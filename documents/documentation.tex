\section{Разработка документации к программному продукту}

\vspace{-0.9cm}

\tocless\subsection{Документация по установке}

Документация по установке проекта algernon и neravarin находится в файле README.md на английском языке.
Выбор языка основывается на том, что исходный код проектов является открытым и распротраняется под свободной лицензией.

\tocless\subsection{Руководство для разработчиков}

Руководство для разработчиков выполнено с помощью встроенного модуля pydoc.

Сама документация описывается в виде docstrings к каждому модулю, классу и функции. Не написать документацию невозможно, так как
проект использует линтер, следяющий за её наличием.

Позже документация автоматически собирается с помощью pydoc в виде html файлов, связанных собой гиперссылками и помещаются в каталог
docs. Дальше они могут использоваться отдельно на машине разработчика.

В дальнейшем планируется перевести документацию на более приемлимый уровень в sphinx и создать возможность получения прямо из проекта,
используя специальный url.
