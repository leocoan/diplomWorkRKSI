\section{Тестирование программного продукта}

Тестирование любого программного продукта необходимо, так как позволяет выявлять ошибки и получать обратную связь от пользователей и/или тестировщиков.
Для наиболее эффективного процесса тестирования выбираются оптимальные способы тестирования, а затем создаются тест-кейсы под уже выбранные методы.

\subsection{Выбор метода обеспечения качества}

Исходя из требований к тестированию необходимо провести его двумя способами:

\begin{enumerate}
    \item создать необходимые автоматические тесты;
    \item вручную протестировать результат работы GraphQL-сервера, созданного в процессе работы.
\end{enumerate}

\subsection{План тестирования}

Требуется провести тестирование работоспособности и качество
реализации проекта «configCastle». В рамках плана следует перечислить
программные составляющие и отдельные составляющие их функционала,
подлежащие оценке качества, подходы и виды тестирования, используемые в
данном процессе и график тестирования.

\subsubsection{Программные составляющие, подлежающие тестированию}

Следует провести тестирование следующих составляющих приложения:

\begin{itemize}
    \item возможность приложения работать базы данных, то есть записывать и получать данные;
    \item ответы GraphQL-сервера, предусмотренных техническим заданием, используя окружение GraphiQL.
\end{itemize}

Автотесты исполняются с помощью системы pytest\_aiohttp, используя экосистему py.test.

\subsubsection{Тест-план}

Для корретного тестирования необходимо составить тест-план с где можно отслеживать какие именно тесты необходимо провести, чтобы
убедиться в работоспобности приложения. Тест-план представлен в таблице \ref{dev:test_plan}.

\begin{longtable}[c]{|l|l|}
    \caption{Тест-план}
    \label{dev:test_plan}\\
    \hline
    Идентификатор & Название тесты \\ \hline
    \endfirsthead
    %
    \multicolumn{2}{c}%
    {{\bfseries Table \thetable\ continued from previous page}} \\
    \hline
    Идентификатор & Название тесты \\ \hline
    \endhead
    %
    1             &      Проверка query service(id: \textless{}Number\textgreater{})          \\ \hline
    2             &      Проверка query services          \\ \hline
    3             &      Проверка query file(id: \textless{}Number\textgreater{})          \\ \hline
    4             &      Проверка query files          \\ \hline
    5             &      Проверка mutation createFile          \\ \hline
    6             &      Проверка mutation updateFile          \\ \hline
    7             &      Првоерка mutation deleteFile         \\ \hline
    8             &      Обновление файла с несуществующим id          \\ \hline
\end{longtable}

\subsubsection{Тест-кейсы}

Для правильного тестирования были составлены подробные тест-кейсы или сценария тестирования, описывающий процесс тестирования
каждого созданного запроса. Данные сценарии представлены в таблицах \ref{test:case_1}, \ref{test:case_2}, \ref{test:case_3}, \ref{test:case_4},
\ref{test:case_5}, \ref{test:case_6}, \ref{test:case_7}, \ref{test:case_8}.

\begin{longtable}[c]{|l|l|}
    \caption{Тест-кейс №1}
    \label{test:case_1}\\
    \hline
    Идентификатор & 1                                                                                                \\ \hline
    \endfirsthead
    %
    \multicolumn{2}{l}%
    {{Продолжение таблицы \thetable}} \\
    \hline
    \endhead
    %
    Название                            & Проверка query service(id: \textless{}Number\textgreater{})                                                            \\ \hline
    Приоритет                           & Высокий                                                                                                               \\ \hline
    Дата проведения                     & 22.04.2020                                                                                                            \\ \hline
    Описание                            & \begin{tabular}[c]{@{}l@{}}Тестируется правильность возвращения\\ данных при использовании query service\end{tabular} \\ \hline
    Предусловие                         & Отсутствует                                                                                                           \\
    \pagebreak
    Шаги тестирования &
      \begin{tabular}[c]{@{}l@{}}1) Запустить graphiQL проекта;\\ 2) Использовать следующий запрос:\\      query \{\\           \hspace{2ex}service(id: 1) \{\\                 \hspace{4ex}name\\                 \hspace{4ex}id\\                 \hspace{4ex}data\\           \hspace{2ex}\}\\      \}\end{tabular} \\ \hline
    Ожидаемый результат                 & \begin{tabular}[c]{@{}l@{}}Появление полей в виде ответа по структуре\\ совпадающих с запросом\end{tabular}           \\ \hline
    Постусловие                         & Отсутсвует                                                                                                            \\ \hline
    Фактический результат               & Совпадает с ожидаемым                                                                                                 \\ \hline
    Статус                              & Пройден                                                                                                               \\ \hline
\end{longtable}

\begin{longtable}[c]{|l|l|}
    \caption{Тест-кейс №2}
    \label{test:case_2}\\
    \hline
    Идентификатор & 2                                                                                                \\ \hline
    \endfirsthead
    %
    \multicolumn{2}{l}%
    {{Продолжение таблицы \thetable}} \\
    \hline
    \endhead
    %
    Название                            & Проверка query services                                                            \\
    Приоритет                           & Высокий                                                                                                               \\ \hline
    Дата проведения                     & 22.04.2020                                                                                                            \\ \hline
    Описание                            & \begin{tabular}[c]{@{}l@{}}Тестируется правильность возвращения\\ данных при использовании query services\end{tabular} \\ \hline
    Предусловие                         & Отсутствует                                                                                                           \\ \hline
    Шаги тестирования &
      \begin{tabular}[c]{@{}l@{}}1) Запустить graphiQL проекта;\\ 2) Использовать следующий запрос:\\      query \{\\           \hspace{2ex}services\{\\                 \hspace{4ex}name\\                 \hspace{4ex}data\\           \hspace{2ex}\}\\      \}\end{tabular} \\
    \pagebreak
    Ожидаемый результат                 & \begin{tabular}[c]{@{}l@{}}Появление полей в виде ответа по структуре\\ совпадающих с запросом\end{tabular}           \\ \hline
    Постусловие                         & Отсутсвует                                                                                                            \\ \hline
    Фактический результат               & Совпадает с ожидаемым                                                                                                 \\ \hline
    Статус                              & Пройден                                                                                                               \\ \hline
\end{longtable}

\begin{longtable}[c]{|l|l|}
    \caption{Тест-кейс №3}
    \label{test:case_3}\\
    \hline
    Идентификатор & 3                                                                                                \\ \hline
    \endfirsthead
    %
    \multicolumn{2}{l}%
    {{Продолжение таблицы \thetable}} \\
    \hline
    \endhead
    %
    Название                            & Проверка query file(id: \textless{}Number\textgreater{})                                                            \\ \hline
    Приоритет                           & Высокий                                                                                                               \\ \hline
    Дата проведения                     & 22.04.2020                                                                                                            \\ \hline
    Описание                            & \begin{tabular}[c]{@{}l@{}}Тестируется правильность возвращения\\ данных при использовании query file\end{tabular} \\ \hline
    Предусловие                         & Отсутствует                                                                                                           \\
    Шаги тестирования &
      \begin{tabular}[c]{@{}l@{}}1) Запустить graphiQL проекта;\\ 2) Использовать следующий запрос:\\      query \{\\           \hspace{2ex}file(id: 1)\{\\                 \hspace{4ex}name\\                 \hspace{4ex}id\\                 \hspace{4ex}data\\           \hspace{2ex}\}\\      \}\end{tabular} \\ \hline
    Ожидаемый результат                 & \begin{tabular}[c]{@{}l@{}}Появление полей в виде ответа по структуре\\ совпадающих с запросом\end{tabular}           \\ \hline
    Постусловие                         & Отсутсвует                                                                                                            \\ \hline
    Фактический результат               & Совпадает с ожидаемым                                                                                                 \\ \hline
    Статус                              & Пройден                                                                                                               \\ \hline
\end{longtable}

\begin{longtable}[c]{|l|l|}
    \caption{Тест-кейс №4}
    \label{test:case_4}\\
    \hline
    Идентификатор & 4                                                                                               \\ \hline
    \endfirsthead
    %
    \multicolumn{2}{l}%
    {{Продолжение таблицы \thetable}} \\
    \hline
    \endhead
    %
    Название                            & Проверка query files                                                         \\ \hline
    Приоритет                           & Высокий                                                                                                               \\ \hline
    Дата проведения                     & 23.04.2020                                                                                                            \\ \hline
    Описание                            & \begin{tabular}[c]{@{}l@{}}Тестируется правильность возвращения\\ данных при использовании query files\end{tabular} \\ \hline
    Предусловие                         & Отсутствует                                                                                                           \\ \hline
    Шаги тестирования &
      \begin{tabular}[c]{@{}l@{}}1) Запустить graphiQL проекта;\\ 2) Использовать следующий запрос:\\      query \{\\           \hspace{2ex}files\{\\                 \hspace{4ex}name\\                               \hspace{4ex}data\\           \hspace{2ex}\}\\      \}\end{tabular} \\ \hline
    Ожидаемый результат                 & \begin{tabular}[c]{@{}l@{}}Появление полей в виде ответа по структуре\\ совпадающих с запросом\end{tabular}           \\ \hline
    Постусловие                         & Отсутсвует                                                                                                            \\ \hline
    Фактический результат               & Совпадает с ожидаемым                                                                                                 \\ \hline
    Статус                              & Пройден                                                                                                               \\ \hline
\end{longtable}

\begin{longtable}[c]{|l|l|}
    \caption{Тест-кейс №5}
    \label{test:case_5}\\
    \hline
    Идентификатор & 5                                                                                                \\ \hline
    \endfirsthead
    %
    \multicolumn{2}{l}%
    {{Продолжение таблицы \thetable}} \\
    \hline
    \endhead
    %
    Название                            & Проверка mutation createFile                                                           \\ \hline
    Приоритет                           & Высокий                                                                                                               \\ \hline
    Дата проведения                     & 23.04.2020                                                                                                            \\ \hline
    Описание                            & \begin{tabular}[c]{@{}l@{}}Тестируется правильность возвращения\\ данных при использовании query service\end{tabular} \\ \hline
    Предусловие                         & Наличие файла с id 0                                                                                                           \\
    Шаги тестирования &
      \begin{tabular}[c]{@{}l@{}}1) Запустить graphiQL проекта;\\ 2) Использовать следующий запрос:\\      mutation \{\\           \hspace{2ex}createFile\{input: \{\\ \hspace{6ex} data: \textquotedblleft\{\}\textquotedblright,\\ \hspace{6ex} name: `` test\_file'',\\ \hspace{6ex} configType: DOCKER\_COMPOSE, \\ \hspace{2ex}) \{\\                 \hspace{4ex}name\\                 \hspace{4ex}id\\                 \hspace{4ex}data\\           \hspace{2ex}\}\\      \}\end{tabular} \\ \hline
    Ожидаемый результат                 & \begin{tabular}[c]{@{}l@{}}Появление полей в виде ответа по структуре\\ совпадающих с запросом\end{tabular}           \\ \hline
    Постусловие                         & Отсутсвует                                                                                                            \\ \hline
    Фактический результат               & Файл создался с id 1                                                                                                 \\
    Статус                              & Провален                                                                                                               \\ \hline
\end{longtable}

\begin{longtable}[c]{|l|l|}
    \caption{Тест-кейс №6}
    \label{test:case_6}\\
    \hline
    Идентификатор & 6                                                                                                \\ \hline
    \endfirsthead
    %
    \multicolumn{2}{l}%
    {{Продолжение таблицы \thetable}} \\
    \hline
    \endhead
    %
    Название                            & Проверка mutation updateFile                                                           \\ \hline
    Приоритет                           & Высокий                                                                                                               \\ \hline
    Дата проведения                     & 24.04.2020                                                                                                            \\ \hline
    Описание                            & \begin{tabular}[c]{@{}l@{}}Тестируется правильность возвращения\\ данных при использовании updateFile\\ и обновление файла\end{tabular} \\ \hline
    Предусловие                         & Существование файла с существующим id                                                                                                           \\
    Шаги тестирования &
      \begin{tabular}[c]{@{}l@{}}1) Запустить graphiQL проекта;\\ 2) Использовать следующий запрос:\\      mutation \{\\           \hspace{2ex}updateFile\{input: \{\\ \hspace{6ex} id: 1 \\ \hspace{6ex} data: \textquotedblleft\{\}\textquotedblright,\\ \hspace{6ex} name: `` updated\_file'',\\ \hspace{6ex} configType: DOCKER\_COMPOSE, \\ \hspace{2ex}) \{\\                 \hspace{4ex}name\\                 \hspace{4ex}id\\                 \hspace{4ex}data\\           \hspace{2ex}\}\\      \}\end{tabular} \\ \hline
    Ожидаемый результат                 & \begin{tabular}[c]{@{}l@{}}Появление полей в виде ответа по структуре\\ совпадающих с запросом\end{tabular}           \\ \hline
    Постусловие                         & Отсутствует                                                                                                            \\ \hline
    Фактический результат               & Файл обновился                                                                                                 \\ \hline
    Статус                              & Пройден                                                                                                               \\ \hline
\end{longtable}

\begin{longtable}[c]{|l|l|}
    \caption{Тест-кейс №7}
    \label{test:case_7}\\
    \hline
    Идентификатор & 7                                                                                                \\ \hline
    \endfirsthead
    %
    \multicolumn{2}{l}%
    {{Продолжение таблицы \thetable}} \\
    \hline
    \endhead
    %
    Название                            & Проверка mutation deleteFile                                                           \\ \hline
    Приоритет                           & Высокий                                                                                                               \\ \hline
    Дата проведения                     & 25.04.2020                                                                                                            \\ \hline
    Описание                            & \begin{tabular}[c]{@{}l@{}}Тестируется правильность возвращения\\ данных при использовании mutation deleteFile\end{tabular} \\ \hline
    Предусловие                         & Файл с id 1                                                                                                           \\
    Шаги тестирования &
      \begin{tabular}[c]{@{}l@{}}1) Запустить graphiQL проекта;\\ 2) Использовать следующий запрос:\\      mutation \{\\           \hspace{2ex}deleteFile(id: 1) \{\\                 \hspace{4ex}name\\                 \hspace{4ex}id\\                 \hspace{4ex}data\\           \hspace{2ex}\}\\      \}\end{tabular} \\ \hline
    Ожидаемый результат                 & \begin{tabular}[c]{@{}l@{}}Появление полей в виде ответа по структуре\\ совпадающих с запросом\end{tabular}           \\ \hline
    Постусловие                         & Файл должен удалится                                                                                                            \\ \hline
    Фактический результат               & Файл удалён и появились указанные поля                                                                                                 \\ \hline
    Статус                              & Пройден                                                                                                               \\ \hline
\end{longtable}

\begin{longtable}[c]{|l|l|}
    \caption{Тест-кейс №8}
    \label{test:case_8}\\
    \hline
    Идентификатор & 8                                                                                                \\ \hline
    \endfirsthead
    %
    \multicolumn{2}{l}%
    {{Продолжение таблицы \thetable}} \\
    \hline
    \endhead
    %
    Название                            & Обновление файла с несуществующим id    \\ \hline
    Приоритет                           & Высокий                                                                                                               \\ \hline
    Дата проведения                     & 25.04.2020                                                                                                            \\ \hline
    Описание                            & \begin{tabular}[c]{@{}l@{}}Тестируется невозможность\\ изменения файла с несуществующим id\end{tabular} \\ \hline
    Предусловие                         & Файл с id 1                                                                                                           \\
    Шаги тестирования &
      \begin{tabular}[c]{@{}l@{}}1) Запустить graphiQL проекта;\\ 2) Использовать следующий запрос:\\      mutation \{\\           \hspace{2ex}deleteFile(id: 666) \{\\                 \hspace{4ex}name\\                 \hspace{4ex}id\\                 \hspace{4ex}data\\           \hspace{2ex}\}\\      \}\end{tabular} \\ \hline
    Ожидаемый результат                 & \begin{tabular}[c]{@{}l@{}}Выброс исключения: ``File does not exist''\end{tabular}           \\ \hline
    Постусловие                         & Отсутствует                                                                                                            \\ \hline
    Фактический результат               & Выброс исключения ``File does not exist''                                                                                                 \\ \hline
    Статус                              & Пройден                                                                                                               \\ \hline
\end{longtable}


\subsection{Баг-репорты}

В процессе тестирования были выявлени два бага общая информациях о которых привидена в таблице \ref{test:bug_all}.

\begin{longtable}[c]{|l|l|l|l|}
    \caption{Зафиксированные баги}
    \label{test:bug_all}\\
    \hline
    № & Название тесты                                                                                                                       & Критичность & Решен \\ \hline
    \endfirsthead
    %
    \multicolumn{4}{c}%
    {{\bfseries Table \thetable\ continued from previous page}} \\
    \hline
    № & Название тесты                                                                                                                       & Критичность & Решен \\ \hline
    \endhead
    %
    1 & \begin{tabular}[c]{@{}l@{}}Не создаётся первый\\ файл в базе данных\end{tabular}                                                     & Высокая     &   \checkmark    \\ \hline
    2 & \begin{tabular}[c]{@{}l@{}}Периодически GraphQL\\ при обновлении файла \\ возвращает None, однако\\ файл обновляется.\end{tabular}   & Низкая      &   \checkmark    \\ \hline
\end{longtable}

Первый баг зафиксиорован в репозитории проекта, где он также помечен как решённый.
