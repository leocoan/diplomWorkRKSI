\section*{\centersection{Введение}}
\addcontentsline{toc}{section}{Введение}

В разработке больших программных продуктов, а в
особенности тех, что используют сервисно-подобную
архитектуру, сегодня всё чаще используется система контейнеризации Docker.

Docker – это программная платформа для быстрой разработки, тестирования
и развертывания приложений как на промышленном сервере, так и на устройстве
каждого разработчика. Docker упаковывает ПО в стандартизованные блоки,
которые называются контейнерами. Каждый контейнер включает все необходимое для
работы приложения: библиотеки, системные инструменты, код и среду исполнения.
Благодаря Docker можно быстро развертывать и масштабировать приложения в любой
среде и сохранять уверенность в том, что код будет работать.

В приложениях, основанных на микросервисной архитектуре, то есть архитектуре программного
обеспечения, ориентированной на взаимодействие насколько это возможно небольших,
слабо связанных и легко изменяемых модулей, очень важно установить правила взаимодействия
микросервисов между собой. В случае с контейнеризацией при помощи Docker, для этого
существует система docker-compose, которая использует файл docker-compose.yaml, в котором
описаны все настройки приложения, основанного на независимых модулях.

Написание данного файла вручную часто связано с неудобствами и возможностью ошибиться
и, тем самым подвергнуть всю инфраструктуру проекта различным рискам.

В рамках подготовки к написанию выпускной квалификационной работы была озвучена
идея создания онлайн сервиса, способного облегчить, структурировать и автоматизировать
работу над написанием docker-compose.yaml и минимизировать вероятность получения неудворительного
результата из-за невнимательности или неопытности разработчика, а также иных обстоятельств.

Дополнительно, пользователю можно будет предоставить возможность хранить множество
конфигураций под разные проекты, делиться ими с членами команды, а также использовать
наработки других разработчиков для облегчения своей задачи и развития технологии контейнеризации.

Для приложения данного типа идеально подходит платформа web, так как этом случае пользователю не требуется:


\begin{itemize}
    \item Хранить тяжеловесное ПО у себя на машине, так как основная часть находится на сервере;
    \item Не понадобится устанавливать обновления;
    \item Сервисом можно будет пользоваться с любого устройства, работающего под любой операционной системой.
\end{itemize}

Более того, web-приложения легко разворачивать и распространять, минуя различные платформы,
распространяющие мобильные или desktop-приложений и их требования к программным продуктам.

Идея была воплощена в онлайн сервисе “Сonfig Castle” - сайте, предоставляющем все вышеперечисленные функции в полном объёме.

В дальнейшем речь пойдет о всех этапах проектирования и тестирования данного сервиса.

