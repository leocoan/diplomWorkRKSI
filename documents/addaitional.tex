\section*{\hfill ПРИЛОЖЕНИЕ А \\ \hspace*{13.5cm} (обязательное)}
\addcontentsline{toc}{section}{Приложение А}


1 Фрагмет кода модуля mutation\_resolvers:

\begin{Verbatim}
        """Module for mutation."""
        from bson.objectid import ObjectId
        from tartiflette import Resolver

        from algernon.utils.db import 

        @Resolver('Mutation.createFile')
            async def resolve_mutation_create_file(parent, args, ctx, system):
            """
            Resolve mutation createFile in GraphqQL API.
            Args:
                parent: initial field in 'execute' method
                args: argument for query
                ctx: context field
                system: information related to the execution
            Returns:
                return created document
            """
            db = ctx['req'].app['db']

            file_id = await return_last_id(db, 'file') + 1

            document = await db['file'].insert_one(
                {
                    'id': file_id,
                    'name': args['input']['name'],
                    'data': args['input']['data'],
                    'configType': args['input']['configType'],
                },
            )

            return await db['file'].find_one(
                {'_id': ObjectId(document.inserted_id)}
            )

        @Resolver('Mutation.deleteFile')
        async def resolve_mutation_delete_file(parent, args, ctx, system):
            """
            Resolve mutation deleteFile in GraphqQL API.
            Args:
                parent: initial field in 'execute' method
                args: argument for query
                ctx: context field
                system: information related to the execution
            Raises:
                Exception: If 'id' of document not found
            Returns:
                Removed document
            """
            db = ctx['req'].app['db']

            document = await db.file.find_one({'id': args['id']})

            if document is None:
                raise Exception('File does not exist.')

            db.file.delete_one({'id': args['id']})

            return document

        @Resolver('Mutation.updateFile')
        async def resolve_mutation_update_file(parent, args, ctx, system):
            """
            Resolve mutation updateFile in GraphqQL API.
            Args:
                parent: initial field in 'execute' method
                args: argument for query
                ctx: context field
                system: information related to the execution
            Raises:
                Exception: If 'id' of document not found
                Exception: If new data not input
            Returns:
                Updated document
            """
            db = ctx['req'].app['db']

            args_input = args['input']

            doc_base = {
                'name': None,
                'configType': None,
                'data': None,
            }

            doc_base['name'] = args_input.get('name')
            doc_base['configType'] = args_input.get('configType')
            doc_base['data'] = args_input.get('data')

            if all(file_value is None for file_value in doc_base.values()):
                raise Exception(
                    "You should provide at least one value: 'name'" +
                    "'type' or 'data'",
                )

            doc_base = {
                key: input_value for key, input_value in doc_base.items()
                if input_value is not None
            }

            if await db.file.find_one({'id': args_input['id']}) is None:
                raise Exception('File does not exist.')

            await db.file.update_one(
                {'id': args_input['id']},
                {'$set': doc_base},
            )

            return await db.file.find_one({'id': args_input['id']})
\end{Verbatim}

2. Фрагмет кода модуля query\_resolvers:

\begin{Verbatim}
        """Query resolovers for API."""
        from tartiflette import Resolver

        from algernon.utils.db import return_all


        @Resolver('Query.services')
        async def resolve_query_services(parent, args, ctx, system):
            """
            Resolve all services.
            Args:
                parent: initial field in 'execute' method
                args: argument for query
                ctx: context field
                system: information related to the execution
            Returns:
                all objects in collections
            """
            db = ctx['req'].app['db']

            return await return_all(db, 'service')


        @Resolver('Query.service')
        async def resolve_query_service(parent, args, ctx, system):
            """
            Resolve one service.
            Args:
                parent: initial field in 'execute' method
                args: argument for query
                ctx: context field
                system: information related to the execution
            Returns:
                one service
            """
            db = ctx['req'].app['db']

            for document in await return_all(db, 'service'):
                if document['id'] == args['id']:
                    return document
            return None


        @Resolver('Query.files')
        async def resolve_query_files(parent, args, ctx, system):
            """
            Resolve all files in system.
            Args:
                parent: initial field in 'execute' method
                args: argument for query
                ctx: context field
                system: information related to the execution
            Returns:
                all objects in collections
            """
            db = ctx['req'].app['db']

            return await return_all(db, 'file')


        @Resolver('Query.file')
            async def resolve_query_file(parent, args, ctx, system):
            """
            Resolve one file.
            Args:
                parent: initial field in 'execute' method
                args: argument for query
                ctx: context field
                system: information related to the execution
            Returns:
                one file
            """
            db = ctx['req'].app['db']

            for document in await return_all(db, 'file'):
                if document['id'] == args.get('id', None):
                    return document

            return None
\end{Verbatim}
