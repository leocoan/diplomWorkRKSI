\section{Техника безопасности и охрана труда}

\tocless\subsection{Требования охраны труда перед началом работы}

\begin{itemize}
    \item Проверить:
    \begin{enumerate}
        \item корректность естественного освещения;
        \item исправность и корректность электроосвещения в кабинете (не менее 300-500 лк на поверхности стола в зоне размещения документа);
        \item площадь рабочего места (не менее 6 м2);
        \item объём рабочего места (не менее 20 м3);
        \item корректность расстояния между мониторами (между основными поверхностями мониторов – не менее 2 м, между боковыми их поверхностями – не менее 1,2 м);
        \item исправность и корректность рабочего кресла (должно быть с подлокотниками, подъёмно-поворотным, с устройством регулировки хода по высоте в пределах 400-550 мм и углам наклона вперёд-назад в пределах 5-15°).
    \end{enumerate}
    \item Проверить работоспособность ПЭВМ, иных электроприборов, а также средств связи, находящихся в кабинете.
    \item Проветрить помещение кабинета.
    \item Проверить безопасность рабочего места на предмет стабильного положения и исправности мебели, измерительных приборов, инструментов, приспособлений, а также проверить наличие в достаточном количестве расходных материалов.
    \item Уточнить план работы на день и, по возможности, распределить намеченное к исполнению равномерно по времени, с включением 15 мин. отдыха (либо кратковременной смены вида деятельности) через каждые 45 мин. однотипных производственных действий, а также с отведением времени в объёме не менее 30 мин. для приёма пищиориентировочно через 4-4,5 ч. луха, памяти, внимания - вследствие ром для решения тех или иных вопросов производственного хара.
\end{itemize}

\tocless\subsection{Требования охраны труда во время работы}

\begin{itemize}
    \item Соблюдать правила личной гигиены.
    \item Исключить пользование неисправным электроосвещением, неработоспособными ПЭВМ, иными электроприборами, а также средствами связи, находящимися в кабинете.
    \item Поддерживать чистоту и порядок на рабочем месте, не загромождать его бумагами, книгами и т.п.
    \item Соблюдать правила пожарной безопасности.
    \item При 8-часовом рабочем дне следует соблюдать регламентированные (технологические) перерывы:
    \begin{enumerate}
        \item При работе по считыванию информации с экрана ПЭВМ с предварительным запросом и суммарным числом считываемых знаков до 60 000 знаков за смену – 20 мин перерыва через 1,5-2 часа после начала рабочего дня и через 1,5-2 часа после обеденного перерыва; альтернативно – 15 мин через каждый час.
        \item При работе по считыванию информации с экрана ПЭВМ с предварительным запросом и суммарным числом считываемых знаков до 40 000 знаков за смену –  15 мин перерыва через 2 часа после начала рабочего дня и через 1,5-2 часа после обеденного перерыва; альтернативно – 10 мин через каждый час.
        \item При работе творческого характера в режиме диалога с ПЭВМ и продолжительностью работы до 6 ч за смену –  15 мин перерыва через 2 часа после начала рабочего дня и через 2 часа после обеденного перерыва.
    \end{enumerate}
\end{itemize}

\tocless\subsection{Требования охраны труда в аварийных ситуациях}

\begin{itemize}
    \item Не приступать к работе при плохом самочувствии или болезни.
    \item В случае возникновения аварийных ситуаций сообщить о случившемся инженеру по охране труда и технике безопасности или, в его отсутствие, дежурному администратору и далее действовать в соответствии с полученными указаниями, а также:
    \item В случае возникновения пожара руководствоваться соответствующим Планом эвакуации, инструкцией по противопожарной безопасности.
    \item В случае угрозы или в случае возникновения очага опасного воздействия техногенного характера руководствоваться соответствующим Планом эвакуации, инструкцией по организации мер безопасности в случае угрозы или в случае возникновения очага опасного воздействия техногенного характера.
    \item В случае угрозы или в случае приведения в исполнение террористического акта руководствоваться соответствующим Планом эвакуации, инструкцией по организации мер безопасности в случае угрозы или в случае приведения в исполнение террористического акта.
    \item При необходимости следует обратиться за помощью и (или) оказать первую помощь пострадавшим от травматизма.
    \item Оказать всемерное содействие расследованию несчастного случая.
\end{itemize}

\tocless\subsection{Требования охраны труда по окончанию работ}

\begin{itemize}
    \item Проветрить кабинет, закрыть окна.
    \item Привести в порядок рабочее место.
    \item Выключить электроприборы, ПЭВМ.
    \item Выключить электроосвещение.
\end{itemize}

Настоящая Инструкция составлена с соблюдением требований действующего законодательства и производственных нормативов, утверждённых постановлением Министерства труда и социального развития РФ от {\textbf{17.12.2002 №80.}